\documentclass{tccv}
\usepackage[english]{babel}
\usepackage[utf8]{inputenc}
\usepackage[T1]{fontenc}

\usepackage{fontawesome}
\newcommand\SYMBOL[1]{\raisebox{-2pt}{\Large\ding{#1}}}

\begin{document}

\part{Manuel Arto}

% About Me
\section{About Me}

Studente magistrale in Informatica presso a Bologna e sviluppatore Full-stack part-time presso Archeion.\newline
Ho esperienza professionale nello sviluppo software, con interessi specifici nella programmazione backend e nella blockchain.

% ESPERIENZA LAVORATIVA
\section{Esperienza Lavorativa}

\begin{eventlist}

     \item{Feb 2024 - Present}
     {\href{https://archeion.tech/}{Archeion}, Bologna}
     {Full-stack Web3 Developer} \\
     \textbullet~Sviluppata l'intera app mobile MVP per i venditori utilizzando Ionic e Angular in meno di 3 mesi. \\
     \textbullet~Migliorata la creazione di schemi per oltre 10 clienti progettando un renderer personalizzato per schemi JSON. \\
     \textbullet~Progettato un servizio di shortening URL scalabile, garantendo oltre 10 milioni di URL attivi. \\
     \textbullet~Migrata l'app cliente a un'architettura SSR e progettata una cache basata su Redis per la traduzione on-demand delle etichette. \\
     \textbullet~Guidato il prototipo ZONIA per il progetto TrustChain finanziato dall'UE: sviluppato un sistema oracle multi-chain zero-trust per dispositivi IoT. \\
     \textbf{\textsc{NestJS · Redis · Solidity · Angular · WoT}}

     \item{Set 2021 - Set 2023}
     {Net Service, Bologna}
     {Backend Developer} \\
     \textbullet~Gestito una piattaforma di dati legali utilizzata da oltre 40 istituti di credito, mantenendo un uptime del 99,9\%. \\
     \textbullet~Ridotti i costi infrastrutturali dell'80\% migrando da un'architettura monolitica a una multitenant. \\
     \textbullet~Sviluppato e distribuito un sistema composto da 5 microservizi utilizzando Spring Boot e OpenShift. \\
     \textbullet~Elaborato un flusso di dati di oltre 100 milioni di record utilizzando MongoDB, PostgreSQL, ActiveMQ e script ETL. \\
     \textbullet~Raggiunto una copertura del codice del 75\% utilizzando il TDD con JUnit per lo sviluppo di nuove funzionalità. \\
     \textbf{\textsc{Spring Boot · OpenShift · MongoDB · PostgreSQL · ActiveMQ}}

\end{eventlist}

% COMPETENZE
\section{Competenze}

\begin{factlist}

\item{Technical}
     {Python, Java, Solidity, Go, Node.js, Flutter, C++, Android, Docker, Linux, Design, Testing} \\

\item{Soft}
     {Teamwork, Communication, Adaptability, Time Management}

\end{factlist}

% \section{Hobby}

% Calisthenics, Scacchi, Manga, Film e Serie TV

% NUOVA PAGINA
\newpage

% INFORMAZIONI PERSONALI
\begin{keyvaluelist}{personal}
    \item[\faHome] Bologna, Italia
    \item[\faPhone] 388 7833963
    \item[\faEnvelope] \href{mailto:manuelarto01@gmail.com}{manuelarto01@gmail.com}
    \item[\faGithub] \href{https://github.com/manuelarto}{Github}
    \item[\faLinkedin] \href{https://www.linkedin.com/in/manuel-arto-696012203/}{LinkedIn}
\end{keyvaluelist}

% ISTRUZIONE
\section{Istruzione}

\begin{yearlist}

\item[Laurea Magistrale]{2023-2025}
     {Informatica}
     {Università di Bologna}

\item[Laurea Triennale]{2020-2023}
    {Informatica}
    {Università di Bologna}

\end{yearlist}

% PROGETTI PERSONALI
\section{Progetti Personali}

\begin{yearlist}

\item{2023}
     {\href{https://github.com/manuelarto/socialtrustr}{SocialTrustr}}
     {\textbullet~Creato un social media framework basato su blockchain, per fornire tracciabilità e validità dei contenuti condivisi. \newline
     \textbullet~Implementato un meccanismo di incetivi per ridurre la condivisione di fake news. \newline
     \textbf{\textsc{Solidity · Blockchain · Consensus Algorithms}}}
\item{2023}
     {\href{https://github.com/manuelarto/livechat}{Livechat}}
     {\textbullet~Sviluppata un'app sociale full-stack con funzionalità di chat in tempo reale e condivisione della posizione tra amici, ottenendo il massimo dei voti (30L) nel corso 'Applicazioni Mobili'. \newline
     \textbf{\textsc{Python · Flutter · Websocket · MongoDB · JWT}}}
\item{2022}
     {\href{https://github.com/manuelarto/crazyplayer}{CrazyPlayer}}
     {\textbullet~Progettato un giocatore AI in grado di giocare in modo ottimale in tutte le istanze possibili del M,N,K-game. \newline
     \textbullet~Classificato primo in un torneo studentesco con oltre 50 partecipanti. \newline
     \textbf{\textsc{Java · Game Tree · Algorithms · Data Structure}}}

\end{yearlist}


% \section{Corsi}

% \begin{yearlist}

% \item[Udemy]{Maggio 2020}
%      {Flutter\&Dart - Guida Completa}
%      {42h, \href{https://www.udemy.com/certificate/UC-c6f5a32f-babc-42f9-8a0a-6effadf9e7cd/}{link certificato}}

% \item[Youtube]{2020-2023}
%     {Sviluppatore Blockchain, Smart Contract, \& Solidity}
%     {Corso completo dal principiante all'esperto, \href{https://github.com/Cyfrin/foundry-full-course-f23}{github repo}}

% \end{yearlist}

% RISULTATI
\section{Extra}

\begin{yearlist}

\item[]{2019}
     {Olimpiadi Italiane Informatica}
     {Competitive Programming in C++}

\item[]{MAG 2019}
     {Gara Nazionale di Informatica}
     {Classificato 5° a livello nazionale \newline
     Progettazione UML e sviluppo in Java}

\item[]{Ott 2018}
     {Erasmus+, Vilnius Lituania}
     {Lavoro full-time in una startup}

\end{yearlist}

\end{document}
